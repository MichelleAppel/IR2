\section{Lexical Models}

In this section we compare different lexical retrieval models.
We first describe the models that we investigated,
and then discuss our comparison experiment.

\subsection{Models}

%%%%%%%%% It should explain what you have implemented

\subsubsection{TF-IDF}

The tf\textendash idf score is the product of two statistics, 
term frequency and inverse document frequency.
The term frequency indicates how often a given term occurs in a document,
while the inverse document frequency quantifies the specificity of the term.
That is, a term is more specific if it occurs less frequently in the documents of the corpus.  

To calculate the tf-idf score of a document for a given query, 
we sum over the tf-idf scores of the individual words in the query.
We use the following formula to calculate these scores: 

\begin{equation*}
tfidf(t,d) = \log(1 + tf(t,d)) \cdot \log\frac{n}{df(t)}
\end{equation*}

We use $\log(1 + tf(t,d))$ instead of the raw term frequency $tf(t,d)$
to account for the fact that relevance	does not increase proportionally with term	
frequency. The document frequency is calculated as $df(t) = \#\{d:tf(t,d) > 0\}$.


\subsubsection{BM25}

BM25 is a slightly more advanced model compared to tf\textendash idf.
In addition to term frequency and inverse document frequency,
it also has a component that controls for the document length.

To calculate the BM25 score of a document for a given query, 
we sum over the BM25 scores of the individual words in the query.
We use the following formula to calculate these scores: 

\begin{equation*}
BM25(t,d) = \frac{(k + 1) tf(t,d)}{k(1 - b + b \cdot \frac{l_d}{l_{avg}} ) + tf(t,d)} \cdot \log\frac{n}{df(t)} 
\end{equation*}

The values $l_d$ and $l_avg$ represent the document length and the average
document length, respectively.
We use $k = 1.5$ and $b = 0.75$ in our evaluation as commonly used default values.
 
\subsubsection{Language Models}

- Jelinek-Mercer

- Dirichlet prior

- Absolute Discounting

\subsubsection{Positional Language Model}

- implementation: we optimized the performance ...


\subsection{Experimental Setup}

%%% Testdata

%%% Tuning hyper params

%%% Metrics
We use trec eval and compare on ...
- NDCG@10, 
- Mean Average Precision (MAP@1000) 
- Precision@5
- Recall@1000.

\subsection{Optimizing Hyper Parameters}

- We use NDCG@10 as a metric. Why?

Jelinek Mercer:
0.1: 0.450
0.5: 0.468
0.9: 0.468

Dirichlet Prior:
500: 0.484
1000: 0.489
1500: 0.484

Absolute Discounting:
0.1: 0.454
0.5: 0.461
0.9: 0.478

Positional Language Model:

\begin{center}
\begin{table}
\scriptsize
  \begin{tabular}{ r | c | c | c }
                & \thead{Dir. 500} & \thead{Dir. 1000} & \thead{Dir. 1500}  \\ \hline
    \thead{Gaussian} & 0.474 & 0.471 & 0.468  \\ \hline
    \thead{Circle}   & 0.469 & 0.468 & 0.462  \\ \hline
    \thead{Passage}  & \cellcolor{blue!25}0.476 & 0.474 & 0.474  \\ \hline
    \thead{Cosine}   & 0.452 & 0.451 & 0.451  \\ \hline
    \thead{Triangle} & 0.456 & 0.454 & 0.454 \\
    \hline
  \end{tabular}

\vspace{5pt}  
  
  \caption{
     Mean NDCG@10 scores on the validation set for the positional language model,
     using different combinations of kernel functions and dirichlet hyper parameter values. The highlighted cell shows the best combination according to
     this experiment.    
  }
  \label{tbl_plm}
\end{table}
\end{center}


\subsection{Results}

- means
\begin{center}
\begin{table}
\scriptsize
  \begin{tabular}{ r | c | c | c | c | c | c }
                & \thead{TF-IDF} & \thead{BM25} & 
                \thead{JM} & \thead{Dir.} & 
                \thead{Abs. D.} & \thead{PLM} \\ \hline
    \thead{prec@5}      & 0.504 & 0.467 & 0.466 & 0.486 & 0.481 & PLM \\ \hline
    \thead{recall@1000} & 0.983 & 0.983 & 0.983 & 0.983 & 0.983 & PLM \\ \hline
    \thead{map@1000}    & 0.409 & 0.408 & 0.407 & 0.414 & 0.408 & PLM \\ \hline
    \thead{ndcg@10}     & \cellcolor{blue!25}0.490 & 0.479 & 0.465 & 0.486 & 0.473 & PLM \\
    \hline
  \end{tabular}

\vspace{5pt}  
  
  \caption{
     Mean scores on the testset for the following retrieval models:
     TF-IDF,
     BM25,
     Jelinek-Mercer (0.5),
     Dirichlet Prior (1000),
     Absolute Discounting (0.9),
     Positional Language Model (k, mu).
     The highlighted cell shows the maximum score for nDCG@10, our main metric.
  }
  \label{tbl_means}
\end{table}
\end{center}


- pvalues

- manual inspection

\subsection{Discussion}

- Do all methods perform similarly on all queries? Why?

- Is there a single retrieval model that outperforms all other retrieval models (i.e., silver bullet)?